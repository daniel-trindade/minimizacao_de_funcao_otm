%%%%%%%%%%%% STRUCTURE %%%%%%%%%%%%%%%
\documentclass[a4paper,12pt]{article}
\usepackage[T1]{fontenc}
\usepackage[utf8]{inputenc}
\usepackage[brazil]{babel}
\usepackage{lmodern}
\usepackage{setspace}
\usepackage[top=2cm, bottom=2cm, left=2cm, right=2cm]{geometry}
%%%%%%%%%%%%%%%%%%%%%%%%%%%%%%%%%%%%%%

%%%%%%%%%%%%%%%% PAGES STYLE %%%%%%%%%
\usepackage{fancyhdr}
\fancypagestyle{main}{
\renewcommand{\headrulewidth}{0pt}
\fancyhead[RO]{\thepage}
\fancyfoot[CO]{}
}
%%%%%%%%%%%%%%%%%%%%%%%%%%%%%%%%%%%%%%

\usepackage{graphicx}
\usepackage{epstopdf}
\usepackage{subfig}
\usepackage{mathptmx}
\usepackage{changepage}
%\usepackage[alf]{abntex2cite}

%%%%%%%%%%% PDF METADATA %%%%%%%%%%%%%
\usepackage[ pdftitle={RELATÓRIO DE ANALISE DE MINIMIZAÇÃO DE FUNÇÕES POR ALGORITMOS GENETICOS E ENXAME DE PARTICULAS},
pdfsubject={MINIMIZAÇÃO DE FUNÇÕES},
pdfkeywords={minimização, algoritmos, geneticos, enxame, particulas},
hidelinks]{hyperref}
%%%%%%%%%%%%%%%%%%%%%%%%%%%%%%%%%%%%%%

\begin{document}

\onehalfspacing

\thispagestyle{empty}

\setcounter{page}{1}

%%%%%%%%%%%% LOGOS %%%%%%%%%%%%%%%%%%%

\begin{figure}[!ht]

\centering

\subfloat{
\includegraphics[width=2.7cm]{imgs/UFRN.pdf}
\label{UFRN Logo}
}
\hspace{11.09cm}
\subfloat{
\includegraphics[width=2.4cm]{imgs/DCA.pdf}
\label{DCA Logo}
}

%\caption{}
\label{Logos}

\end{figure}

%%%%%%%%%%%%%%% CAPA %%%%%%%%%%%%%%%%%

\vspace{-1cm}

\begin{center}
{\bf{\normalsize UNIVERSIDADE FEDERAL DO RIO GRANDE DO NORTE\\
DEPARTAMENTO DE ENGENHARIA DE COMPUTAÇÃO E AUTOMAÇÃO\\
CURSO DE ENGENHARIA DE COMPUTAÇÃO\\
OTIMIZAÇÃO DE SISTEMAS
}}


\vspace{3.6cm}

{\bf{\large RELATÓRIO DO TRABALHO DE MINIMIZAÇÃO DE FUNÇÕES\\
ATRAVÉS DE ALGORITMOS GENÉTICOS E ENXAME DE PARTÍCULAS\\
}}
\vspace{1.5cm}


\vspace{5.6cm}



\begin{flushright}
\begin{normalsize}
DANIEL BRUNO TRINDADE DA SILVA - 20230093910\\

\end{normalsize}
\end{flushright}


\vspace{6.5cm}

{\large Natal-RN\\
2025.2}
\end{center}

\newpage

%%%%%%%%%%%%%%%  RESUMO %%%%%%%%%%%%%%

\thispagestyle{empty}

\begin{center}
{\large \textbf{RESUMO}}
\end{center}

\vspace{3cm}

\begin{flushleft}

\hspace{4ex}Trata-se da apresentação fiel, breve e concisa dos aspectos mais relevantes do
trabalho, apresentando as ideias essenciais, na mesma progressão e no mesmo
encadeamento que aparecem no texto. O resumo deve apresentar os objetivos, uma
visão geral, ampla e, ao mesmo tempo, clara e objetiva do conteúdo do trabalho.\\
\hspace{4ex}A norma da ABNT recomenda que se use de 150 a 500 palavras, em espaço
simples, e deve-se usar o verbo na voz ativa e na terceira pessoa singular. Logo abaixo,
devem ser colocadas as palavras-chave.\\

\end{flushleft}

\vspace{1.5cm}

\textbf{Palavras-chave:}

\newpage

%%%% LISTA DE ABREVIATURAS E SIGLAS %%

\thispagestyle{empty}

\begin{center}
{\large \textbf{LISTA DE ABREVIATURAS E SIGLAS}}
\end{center}

\vspace{3cm}

\begin{tabular}{ l l }
AG\hspace{1.5cm} & Algoritmo Genético.\\
PSO\hspace{1.5cm} & \textit{Particle Swarm Optimization} (Otimização por Enxame de Partículas)\\

\end{tabular}

\newpage

%%%%%%%%% LISTA DE FIGURAS %%%%%%%%%%%

\thispagestyle{empty}

\begin{center}
\listoffigures
\end{center}

\newpage

%%%%%%%%%%%%%%% SUMÁRIO %%%%%%%%%%%%%%

\thispagestyle{empty}

\begin{center}
\tableofcontents
\end{center}

\newpage

%%%%%%%%%%%%%%% INTRODUÇÃO %%%%%%%%%%%

\thispagestyle{main}

\section{INTRODUÇÃO}

\hspace{4ex}A otimização de funções matemáticas complexas é um desafio central em diversas áreas da engenharia. Quando o espaço de busca é vasto e a função objetivo apresenta múltiplos mínimos locais, métodos analíticos tradicionais tornam-se ineficientes. Nesse contexto, algoritmos metaheurísticos inspirados na natureza destacam-se como soluções robustas para encontrar o ótimo global.

O presente trabalho tem como objetivo implementar e analisar o desempenho de duas das principais técnicas de computação evolucionária e inteligência: o Algoritmo Genético (AG) e a Otimização por Enxame de Partículas (PSO - \textit{Particle Swarm Optimization})1. A meta é minimizar a função objetivo composta definida pela combinação $W29 + W1$, conforme designado na especificação do projeto.

Para além da simples obtenção do mínimo global, este estudo foca na análise comparativa do custo computacional de cada abordagem. Conforme os requisitos propostos, foram mensuradas a quantidade de avaliações da função objetivo e o número de operações aritméticas (multiplicações e divisões) necessárias para a convergência. A estabilidade dos algoritmos também foi avaliada estatisticamente, verificando a taxa de sucesso na convergência em múltiplas execuções independentes.

O desenvolvimento foi realizado utilizando a linguagem Python, escolhida pela sua capacidade de prototipagem rápida e bibliotecas de manipulação numérica, permitindo uma instrumentação precisa do código para a coleta das métricas de desempenho exigidas.

\newpage

%%%%%%%%%% REFERENCIAL TEÓRICO %%%%%%%

\thispagestyle{main}

\section{REFERENCIAL TEÓRICO}

A resolução de problemas de otimização não-linear, especialmente aqueles com múltiplas variáveis e ótimos locais, frequentemente excede a capacidade de métodos analíticos tradicionais baseados em gradiente. Nestes cenários, algoritmos metaheurísticos inspirados na natureza demonstram eficácia superior na exploração do espaço de busca global.

\subsection{Algoritmos Genéticos (AG)} 

Desenvolvidos inicialmente por John Holland na década de 1960 e popularizados por David Goldberg, os Algoritmos Genéticos são técnicas de busca estocástica baseadas nos princípios da seleção natural e genética de populações (HOLLAND, 1975).

O funcionamento do AG baseia-se na evolução de uma população de indivíduos (candidatos à solução) ao longo de gerações. A aptidão (\textit{fitness}) de cada indivíduo determina sua probabilidade de sobrevivência e reprodução, seguindo a premissa Darwiniana de que os mais aptos tendem a propagar suas características (GOLDBERG, 1989).

Os principais operadores que conduzem a busca são:

\begin{itemize}
	\item \textbf{Seleção:} Mecanismo que escolhe os pais para a reprodução. Neste trabalho, utiliza-se o método da Roleta, onde a probabilidade de seleção de um indivíduo é proporcional à sua aptidão relativa na população.
	\item \textbf{Cruzamento (\textit{Crossover}):} Processo de recombinação genética onde dois pais trocam informações para gerar descendentes. Para problemas com variáveis contínuas (codificação real), utilizam-se operadores aritméticos ou mistos, como o BLX-$\alpha$, que permitem explorar o espaço entre os pais (HERRERA et al., 1998).
	\item \textbf{Mutação:} Operador responsável por manter a diversidade genética da população, introduzindo pequenas perturbações aleatórias nos genes, evitando a convergência prematura para mínimos locais (MITCHELL, 1998).
\end{itemize}

\subsection{Otimização por Enxame de Partículas (PSO)}

A Otimização por Enxame de Partículas é um algoritmo metaheurístico proposto por Kennedy e Eberhart em 1995, inspirado no comportamento social de organismos simples, como bandos de pássaros ou cardumes de peixes (KENNEDY; EBERHART, 1995).

Diferente do AG, o PSO não utiliza operadores de evolução (como cruzamento ou mutação). Em vez disso, mantém uma população de partículas que "voam" pelo hiperespaço de busca. Cada partícula ajusta sua trajetória baseando-se em duas experiências:

\begin{enumerate}
	\item Cognitiva ($p_{best}$): A melhor posição encontrada pela própria partícula até o momento.
	\item Social ($g_{best}$): A melhor posição encontrada por qualquer partícula do enxame (vizinhança).
\end{enumerate}

Matematicamente, a velocidade ($v$) e a posição ($x$) de cada partícula $i$ são atualizadas pelas equações:

$$v_{i}(t+1) = w \cdot v_{i}(t) + c_1 \cdot r_1 \cdot (p_{best} - x_{i}(t)) + c_2 \cdot r_2 \cdot (g_{best} - x_{i}(t))$$$$x_{i}(t+1) = x_{i}(t) + v_{i}(t+1)$$

Onde $c_1$ e $c_2$ são coeficientes de aceleração (cognitivo e social) e $r_1, r_2$ são valores aleatórios. O parâmetro $w$ (peso de inércia), introduzido posteriormente por Shi e Eberhart (1998), é crucial para controlar o balanço entre a exploração global (\textit{exploration}) e o refinamento local (\textit{exploitation}).


\newpage

%%%%%%%%%% METODOLOGIA %%%%%%%%%%%%%%%

\thispagestyle{main}

\section{METODOLOGIA}


\hspace{4ex}A metodologia é caracterizada pela explicação minuciosa dos procedimentos
técnicos realizados durante todo o trabalho.

\subsection{Seções}

\subsubsection{Subseções}

\newpage

%%%%%%%%%% RESULTADOS %%%%%%%%%%%%%%%

\thispagestyle{main}

\section{RESULTADOS}


\hspace{4ex}Neste capítulo, são apresentados e descritos os resultados obtidos dos experimentos feitos em laboratório. É importante que todos os gráficos e figuras apresentados neste capítulo estejam bem visíveis e com qualidade boa. Este é o capítulo mais importante do trabalho, pois é nele que o aluno irá descrever todos os resultados e observações obtidos no experimento.

\subsection{Seções}

\subsubsection{Subseções}

\newpage

%%%%%%%%%% CONCLUSÃO %%%%%%%%%%%%%%%

\thispagestyle{main}

\section{CONCLUSÃO}


\hspace{4ex}A conclusão, além de guardar uma proporção relativa ao tamanho do trabalho,
deve guardar uma proporcionalidade também quanto ao conteúdo. Não deve conter
assuntos desnecessários, nem exageros numa linguagem excessivamente técnica e
rebuscada. A conclusão deve dar respostas às questões do trabalho, correspondente aos
objetivos propostos. Deve ser breve, podendo, se necessário, apresentar sugestões para
pesquisas futuras.

\newpage

%%%%%%%% REFERÊNCIAS %%%%%%%%%%%%%%%%%

% Referências bibliogáficas (geradas automaticamente)
\addcontentsline{toc}{chapter}{Referências bibliográficas}
\bibliography{bib/bibliografia}

\appendix

%Apêndice A
\include{apendice}

\end{document}